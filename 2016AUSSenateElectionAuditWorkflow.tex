\documentclass[]{article}
\usepackage{lmodern}
\usepackage{amssymb,amsmath}
\usepackage{ifxetex,ifluatex}
\usepackage{fixltx2e} % provides \textsubscript
\ifnum 0\ifxetex 1\fi\ifluatex 1\fi=0 % if pdftex
  \usepackage[T1]{fontenc}
  \usepackage[utf8]{inputenc}
\else % if luatex or xelatex
  \ifxetex
    \usepackage{mathspec}
  \else
    \usepackage{fontspec}
  \fi
  \defaultfontfeatures{Ligatures=TeX,Scale=MatchLowercase}
\fi
% use upquote if available, for straight quotes in verbatim environments
\IfFileExists{upquote.sty}{\usepackage{upquote}}{}
% use microtype if available
\IfFileExists{microtype.sty}{%
\usepackage{microtype}
\UseMicrotypeSet[protrusion]{basicmath} % disable protrusion for tt fonts
}{}
\usepackage[unicode=true]{hyperref}
\hypersetup{
            pdfborder={0 0 0},
            breaklinks=true}
\urlstyle{same}  % don't use monospace font for urls
\IfFileExists{parskip.sty}{%
\usepackage{parskip}
}{% else
\setlength{\parindent}{0pt}
\setlength{\parskip}{6pt plus 2pt minus 1pt}
}
\setlength{\emergencystretch}{3em}  % prevent overfull lines
\providecommand{\tightlist}{%
  \setlength{\itemsep}{0pt}\setlength{\parskip}{0pt}}
\setcounter{secnumdepth}{0}
% Redefines (sub)paragraphs to behave more like sections
\ifx\paragraph\undefined\else
\let\oldparagraph\paragraph
\renewcommand{\paragraph}[1]{\oldparagraph{#1}\mbox{}}
\fi
\ifx\subparagraph\undefined\else
\let\oldsubparagraph\subparagraph
\renewcommand{\subparagraph}[1]{\oldsubparagraph{#1}\mbox{}}
\fi

\title{\protect\hypertarget{_kijxneej980z}{}{}

\protect\hypertarget{_1t5wdgb8h9yb}{}{}2016 AUS Senate Audit Workflow}
\date{}

\begin{document}
\maketitle

\subsection{}\label{section}

\section{1. Overview}\label{overview}

This document summarizes the election workflow used to audit the 2016
Australian Senate Election.

There are many ways that error can alter the outcome of elections. The
Australian Senate Election has complex voting rules and delicate
dependence on software, including OCR, which make the outcome
particularly vulnerable to small errors. Auditing electronic results
against the underlying voter-marked paper records can help determine the
level of confidence warranted.

\section{2. Workflow}\label{workflow}

The audits we propose are incremental, in a series of rounds, starting
with an initial sample of ballots. Tests determine whether the sample
provides strong evidence that the electoral outcome is correct. If not,
the sample is augmented and the process repeats, round after round,
until the outcome is confirmed, until all ballots have been examined, or
until a preset limit on time or audit resources has been met.

Each time a sample is enlarged, randomly selected paper ballots are
retrieved and interpreted manually (hand-to-eye), and those manual
interpretations are recorded electronically. Those new electronic
records are the basis of the audit computations.

\subsection{2.1 Stage 1: Planning}\label{stage-1-planning}

The in-charge of the audit will identify the individuals to participate
in performing the audit.

They will meet to review and refine the steps to be performed, the
observation of each step, the allocation of resources, the time-table
involved, any physical security protocols to be followed, the funding
necessary for any expenses, the reporting protocols to be followed, the
maintenance of chain of custody over the paper ballots, the
determination as to which computers (laptops) will be employed and how
they will initialized with the appropriate software, what procedures
will be followed for handling exceptions or errors, etc.

\subsection{2.2 Stage 2: Choosing Ballots to be
Audited}\label{stage-2-choosing-ballots-to-be-audited}

Paper ballots are selected at random for transcription and auditing,
using a pseudo-random number generator seeded with a publicly-chosen
random value. The output of this stage is a set of listing specified
which paper ballots will be audited in each round of the audit.

\subsubsection{2.2.1 Generate Random Seed
Publicly}\label{generate-random-seed-publicly}

The seed in the pseudo-random number generator is set using a physical
source of randomness, by rolling 24 10-sided dice. This ensures that the
ballots to be selected are truly unpredictable, but that the process is
repeatable, starting from the archived seed. The dice-rolling should be
a public ceremony, observed by a number of scrutineers from the various
parties.

\subsubsection{\texorpdfstring{2.2.2 Running \emph{Sampler -\/- Choosing
Random
Numbers}}{2.2.2 Running Sampler -\/- Choosing Random Numbers}}\label{running-sampler----choosing-random-numbers}

To choose the paper ballots to review, we run \emph{sampler.py. }

We run sampler with the following parameters:

\begin{enumerate}
\def\labelenumi{\arabic{enumi}.}
\item
  \begin{quote}
  Election Id: ``2016 AUS Senate Election {[}STATE{]} Audit Round
  {[}\#{]}''
  \end{quote}
\item
  \begin{quote}
  Random seed: 24 decimal digits generated in step 2.1.1
  \end{quote}
\item
  \begin{quote}
  Output range: 2 to number of rows in
  \emph{aec-senate-formalpreferences-STATE.csv}.
  \end{quote}
\item
  \begin{quote}
  Duplicates ok: no
  \end{quote}
\item
  \begin{enumerate}
  \def\labelenumii{\alph{enumii}.}
  \item
    \begin{quote}
    Expanded version of sample: No if first round, yes for subsequent
    round. If non-first round enter the total cumulative number of
    ballots previously sampled.
    \end{quote}
  \item
    \begin{quote}
    How many ballots: \textbf{1500}
    \end{quote}
  \end{enumerate}
\item
  \begin{quote}
  Output file: ``2016\_AUS\_Senate\_Election\_{[}STATE{]}\_Audit
  Round\_{[}\#{]}.txt''
  \end{quote}
\end{enumerate}

\subsubsection{2.2.3 Listing Ballots}\label{listing-ballots}

At each round after sampler.py has been run, a csv will be generated
named selected\_ballots\_round\_\# to facilitate retrieval of ballots by
election officials. This CSV will be generated by running
sampleballotgenerator.py using the sampler.py output and
\emph{aec-senate-formalpreferences-STATE.csv} as input. The generated
csv named \emph{selected\_ballots\_round\_\#.csv} can be printed with 47
rows per page. The original preference information is \emph{not} listed
here, to prevent bias on the part of the auditors.

The CSV header \emph{selected\_ballots\_round\_\#.csv} is as follows:

\emph{ElectorateNm, VoteCollectionPointNm, VoteCollectionPointId,
BatchNo, PaperNo, Preferences}

\subsection{2.3 Stage 3: Obtaining and transcribing the selected paper
ballots}\label{stage-3-obtaining-and-transcribing-the-selected-paper-ballots}

The selected\_ballots\_round\_\#.csv will be sent to the officials for
the counting step. At least two election officials are required at this
stage to verify correctness at each step.

\subsubsection{2.3.1 Retrieving ballots}\label{retrieving-ballots}

At this step a pair of election officials will take the a page of
\emph{SelectedBallots.csv} and retrieve the ballots listed on the sheet.
They will take the sheet and retrieve ballots in the following manner.
For each ballot in the list, both officials will confirm its placement
in the pile. When a ballot is extracted, officials will place a yellow
sticky note in the batch with the appropriate ballot information:
(District, batch, paper no) and place a pink sticky note with identical
information on the extracted ballot.

These ballots will be placed in order in a ballot box and brought over
for manual entry, with the corresponding sheet. Batch's open

\protect\hypertarget{_lba7oli3yiio}{}{}2.3.2 Ballot Entry

The preferences observed on the obtained paper ballots will be entered
into \emph{selected\_ballots\_round\_\#.csv.} From the box of collected
ballots, each ballot will be read and the preferences reinterpreted by
both election officials directly from the paper ballots. Once the
election officials agree on the preferences, they will enter the
preferences in the \emph{selected\_ballots\_round\_\#.csv} preference
field as a list of candidate IDs.

\subsubsection{2.3.3 Entry review}\label{entry-review}

Once the all the ballots have been entered into the
\emph{selected\_ballots\_round\_\#.csv.} The form will be sent to the
audit group to run the audits on the new records.

\subsection{2.4 Running Audits}\label{running-audits}

\subsubsection{2.4.1 Generating Audit Data}\label{generating-audit-data}

From the filled out \emph{selected\_ballots\_round\_\#.csv and
aec-senate-formalpreferences-STATE.csv} generate\_recount\_data.py will
be run to create a new CSV with the recount preferences and whether the
initial and recounted preferences match. The header for the

\emph{aec\_senate\_formal\_preferences\_audit\_round\_\#\_.csv} will be
as follows\emph{:}

\emph{ElectorateNm, VoteCollectionPointNm, VoteCollectionPointId,
BatchNo, PaperNo, Preferences, RecountPreferences, Match}

\subsubsection{2.4.1 Concatenate Data}\label{concatenate-data}

From the new
\emph{aec\_senate\_formal\_preferences\_audit\_round\_\#\_.csv} the data
will be appended to
\emph{aec\_senate\_formal\_preferences\_audit\_aggregate.csv} to form a
combined data set for the audit round.

\subsubsection{2.4.2 Run Audit}\label{run-audit}

From the \emph{aec\_senate\_formal\_preferences\_audit\_aggregate.csv,}
various audits will be run to determine whether to proceed with another
round.

\section{3 Design Considerations / Open
Questions}\label{design-considerations-open-questions}

\subsection{3.1 Number of Ballots}\label{number-of-ballots}

1500 ballots were chosen as the size of ballots per round.

\subsection{3.2 Current Tabulation
System}\label{current-tabulation-system}

Although the current tabulation system operates of of scans, we require
hand to eye interpretations of ballots as the most accurate means of
verifying the outcome.

\section{4 Requirements}\label{requirements}

\subsection{4.1 Ballot Selection}\label{ballot-selection}

For ballot selection, two distinct people and some ten-sided dice are
needed. Also, the following software andtools are required: Python,
sampler.py, \emph{aec-senate-formalpreferences.csv}. , and
ballot\_selector.py

\subsection{4.2 Ballot Counting}\label{ballot-counting}

Each ballot counting stage requires 2 people per entry or ballot
retrieval step, aec\_senate\_formal\_preferences\_audit\_round\_\#\_.csv
and selected\_ballots.csv.

\subsection{4.3 Running Audit}\label{running-audit}

In order to run the audit the following software and tools are required:

The audit code here:
\href{https://github.com/ron-rivest/2016-aus-senate-audit}{\emph{https://github.com/ron-rivest/2016-aus-senate-audit}}
needs to be installed with the dividebatur code in the root directory of
the 2016-aus-senate-audit directory:
\href{https://github.com/grahame/dividebatur/tree/master/dividebatur}{\emph{https://github.com/grahame/dividebatur/tree/master/dividebatur}}
. These tools require Python 3, git lfs, and numpy.

These tools are currently being refactored, and access to an AWS image
with the libraries and bash run scripts can be requested by emailing
\href{mailto:zperumal@mit.edu}{\emph{zperumal@mit.edu}}.

\section{5 Future Work }\label{future-work}

\end{document}
